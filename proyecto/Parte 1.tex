%%%%%%%%%%%%%%%%%%%%%%%%%%%%%%%%%%%%%%%%%
% Programming/Coding Assignment
% LaTeX Template
%
% This template has been downloaded from:
% http://www.latextemplates.com
%
% Original author:
% Ted Pavlic (http://www.tedpavlic.com)
%
% Note:
% The \lipsum[#] commands throughout this template generate dummy  text
% to fill the template out. These commands should all be removed when 
% writing assignment content.
%
% This template uses a Perl script as an example snippet of code, most other
% languages are also usable. Configure them in the "CODE INCLUSION 
% CONFIGURATION" section.
%
%%%%%%%%%%%%%%%%%%%%%%%%%%%%%%%%%%%%%%%%%

%----------------------------------------------------------------------------------------
%	PACKAGES AND OTHER DOCUMENT CONFIGURATIONS
%----------------------------------------------------------------------------------------

\documentclass{article}

\usepackage{fancyhdr} % Required for custom headers
\usepackage{lastpage} % Required to determine the last page for the footer
\usepackage{extramarks} % Required for headers and footers
\usepackage[usenames,dvipsnames]{color} % Required for custom colors
\usepackage{graphicx} % Required to insert images
\usepackage{listings} % Required for insertion of code
\usepackage{courier} % Required for the courier font
\usepackage{multirow}
\usepackage{hyperref}


% Margins
\topmargin=-0.45in
\evensidemargin=0in
\oddsidemargin=0in
\textwidth=6.5in
\textheight=9.0in
\headsep=0.25in

\linespread{1.1} % Line spacing

%----------------------------------------------------------------------------------------
%	CODE INCLUSION CONFIGURATION
%----------------------------------------------------------------------------------------

\definecolor{MyDarkGreen}{rgb}{0.0,0.4,0.0} % This is the color used for comments
\lstloadlanguages{c} % Load Perl syntax for listings, for a list of other languages supported see: ftp://ftp.tex.ac.uk/tex-archive/macros/latex/contrib/listings/listings.pdf
\lstset{language=[sharp]c, % Use Perl in this example
        frame=single, % Single frame around code
        basicstyle=\small\ttfamily, % Use small true type font
        keywordstyle=[1]\color{Blue}\bf, % Perl functions bold and blue
        keywordstyle=[2]\color{Purple}, % Perl function arguments purple
        keywordstyle=[3]\color{Blue}\underbar, % Custom functions underlined and blue
        identifierstyle=, % Nothing special about identifiers                                         
        commentstyle=\usefont{T1}{pcr}{m}{sl}\color{MyDarkGreen}\small, % Comments small dark green courier font
        stringstyle=\color{Purple}, % Strings are purple
        showstringspaces=false, % Don't put marks in string spaces
        tabsize=5, % 5 spaces per tab
        %
        % Put standard Perl functions not included in the default language here
        morekeywords={rand},
        %
        % Put Perl function parameters here
        morekeywords=[2]{on, off, interp},
        %
        % Put user defined functions here
        morekeywords=[3]{test},
       	%
        morecomment=[l][\color{Blue}]{...}, % Line continuation (...) like blue comment
        numbers=left, % Line numbers on left
        firstnumber=1, % Line numbers start with line 1
        numberstyle=\tiny\color{Blue}, % Line numbers are blue and small
        stepnumber=5 % Line numbers go in steps of 5
}

\newcommand{\horrule}[1]{\rule{\linewidth}{#1}}

% Creates a new command to include a perl script, the first parameter is the filename of the script (without .pl), the second parameter is the caption
\newcommand{\perlscript}[2]{
\begin{itemize}
\item[]\lstinputlisting[caption=#2,label=#1]{#1.cs}
\end{itemize}
}

\begin{document}

\begin{tabular}{l l}
\multirow{5}{*}{\includegraphics[width=2cm]{../recursos/logo.png}} & Universidad del Istmo de Guatemala \\
 & Facultad de Ingenieria \\
 & Ing. en Sistemas \\
 & Curso libre configuraci\'on II: Blockchains \\
 & Prof. Ernesto Rodriguez - \href{mailto:erodriguez@unis.edu.gt}{erodriguez@unis.edu.gt} \\
\end{tabular}
\\\\\\

\begin{center}
        \horrule{0.5pt}
        \huge{Proyecto: Entrega Parcial \#1} \\
        \large{Fecha de entrega: 11 de Agosto, 2019 - 11:59pm} \\
        {Entrega preliminar: 1 de Agosto, 2019 - periodo de clase}
        \horrule{1pt}
\end{center}

\section*{Introducci\'on}

El objetivo de este proyecto es que el estudiante ponga en practica lo aprendido
durante el curso de ``Ingenieria de Software'' para aplicar sus conocimientos sobre
\emph{Blockchains} que seran aprendidos durante este curso.
\\\\
El estudiante tendra que resolver un problema de su eleccion utilizando Blockchains.
Para ello el estudinate podra:
\begin{itemize}
        \item Crear un Blockchain especializado para ese problema
        \item Utilizar un Blockchain existente para resolver el problema
        \item Extender un Blockchain existente para resolver el problema
\end{itemize}

El problema que se trabajara no debe ser estrictamente un problema original
del estudiante, puede tomar un problema existente y resolverlo. Se motiva al
estudiante que escoga un problema de tal manera que su soluci\'on resulte en
un aporte a la comunidad de software libre.
\\\\
Este trabajo se puede hacer de forma individual o en grupos de tama\~no
arbitrario. Debe tomar en cuenta que las expectativas seran m\'as altas
si el trabajo se lleva a cabo en grupos de mayor tama\~no.
\\\\
Este trabajo se entregara en 4 etapas, cada etapa debe mostrar avances
en el proyecto. Se puede considerar cada entrega como un ``sprint'', seg\'un
las metodologias agiles, sin embargo, seran calendarizadas en las semanas
de examenes parciales ya que estas entregas seran la nota del examen parcial
y final.

\section*{Descripci\'on de la entrega \#1}
El estudiante debe preparar un documento con el siguiente
contenido:
\begin{itemize}
        \item Integrantes del grupo
        \item Descripci\'on del problema que se trabajara
        \item Propuesta para solucionar el problema
        \item Herrameintas y technologias que se utilizaran el problema
        \item Metodo para evaluar su soluci\'on y avances
        \item {
                Cronograma:
                \begin{itemize}
                        \item{
                                Establecer que sera entregado
                                en la entrega \#2, \#3, y entrega final
                        }
                        \item{
                                Establecer como sera evaluada cada
                                entrega.
                        }
                        \item{
                                Establecer que se presentara durante
                                cada entrega.
                        }
                \end{itemize}
        }
\end{itemize}

\section*{Entrega y Calificaci\'on}

El estudiante debe hacer una entrega preliminar seguido de una entrega
final de este documento. La entrega preliminar tendra un valor de
5\% de la zona y la entrega tendra un valor de 10\% de la zona.
Durante la entrega preliminar, el profesor les estara dando retroalimentaci\'on
del documento. Es muy importante que se participe en esta entrega ya que
eso le permitira establecer parametros razonables para las siguientes entregas
y la entrega final.

\end{document}